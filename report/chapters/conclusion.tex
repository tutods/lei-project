\chapter{Conclusão}

\begin{flushright}
	\begin{quotebox50}
		\large
		“The only way to learn a new programming language is by writing programs in it.”

		\tcblower
		Dennis Ritchie
	\end{quotebox50}
\end{flushright}

Apesar do projeto não se encontrar finalizado, foi possível cooperar e adquirir novos conhecimentos em ambas as vertentes (\textit{backoffice} e \textit{frontoffice}) do mesmo. Além disso, foi ainda possível praticar e melhorar conhecimentos já existentes, aprofundando todo o conhecimento sobre a biblioteca \textbf{React}. Porém, é necessário realçar que todos os objetivos foram atingidos, sendo estes, de uma forma geral, as funcionalidades e \textit{\glslinkUnder{layout}{layout}} de inicio de sessão e criação de conta, bem como o \textit{boilerplate} inicial do projeto, seguindo os \textit{mockups} desenvolvidos para este.

Com o projeto foi ainda possível conhecer novas ferramentas e investigar algumas destas durante os tempos livres, como por exemplo o uso de \textbf{Styled Components} para definir o estilo da aplicação/projeto.

É importante referir que apesar de determinadas funcionalidades não terem sido implementadas, surgiu o interesse pela descoberta de soluções para implementações das mesmas, bem como a descoberta de novas ferramentas e \textit{\glslinkUnder{packages}{packages}}.

% ==> Extra Projects <== %
\section{Formações Adicionais}

De forma a conseguir adquirir novos e aprofundar conhecimentos, foram realizadas algumas formações adicionais, complementares ao estágio (por iniciativa própria). A lista que se segue apresenta projetos realizados durante estas formações, bem como as respetivas ligações para o código desenvolvido nestas.

\clearpage

\begin{minipage}[t]{0.45\textwidth}
	\begin{itemize}
		\item \textbf{Next Level Week 04}
			\begin{itemize}
				\item \href{http://nextlevelweek.com/}{Informações};
				\item \textbf{Repositório:} \href{https://github.com/TutoDS/nlw04-react}{GitHub};
				\item \textbf{\textit{Deploy}:} \href{https://move-it-tutods.vercel.app}{Vercel};
			\end{itemize}

		\item \textbf{\textit{ReactJS Challenge~\textemdash~Slack Clone}}
			\begin{itemize}
				\item \href{https://www.youtube.com/channel/UCqrILQNl5Ed9Dz6CGMyvMTQ}{Canal do YouTube}
				\item \textbf{Repositório:} \href{https://github.com/TutoDS/reactjs-slack-clone-challenge}{GitHub};
				\item \textbf{\textit{Deploy}:} \href{https://slack-clone-challenge-c35ca.web.app/}{Firebase};
			\end{itemize}

			\item \textbf{Twitter \textit{UI Clone}}
			\begin{itemize}
				\item \href{https://www.youtube.com/watch?v=K-8z_4xvT3o}{Vídeo}
				\item \textbf{Repositório:} \href{https://github.com/TutoDS/twitter-ui-clone}{GitHub};
				\item \textbf{\textit{Deploy}:} \href{https://twitter-clone-tutods.netlify.app/}{Netlify}
			\end{itemize}
	\end{itemize}
\end{minipage}
\begin{minipage}[t]{0.45\textwidth}
	\begin{itemize}
		\item \textbf{Next Level Week 05}
			\begin{itemize}
				\item \href{http://nextlevelweek.com/}{Informações};
				\item \textbf{Repositório:} \href{https://github.com/TutoDS/nlw05-react}{GitHub};
				\item \textbf{\textit{Deploy}:} \href{https://podcastr-tutods.vercel.app/}{Vercel}
			\end{itemize}

		\item \textbf{LinkedIn \textit{UI Clone}}
			\begin{itemize}
				\item \href{https://www.youtube.com/watch?v=xP3cxbDUtrc}{Vídeo}
				\item \textbf{Repositório:} \href{https://github.com/TutoDS/reactjs-linkedin-clone}{GitHub}
				\item Por Concluir
			\end{itemize}

		\item \textbf{TypeGraphQL} - Code/drop \#74
			\begin{itemize}
				\item \href{https://www.youtube.com/watch?v=qMc5A5-Ktuw}{Vídeo}
				\item \textbf{Repositório:} \href{https://github.com/TutoDS/typegraphql-code-drops-74}{GitHub}
				\item \textbf{Temas Principais:} GraphQL e TypeGraphQL
			\end{itemize}
	\end{itemize}
\end{minipage}

\vspace{10pt}

Nos projetos apresentados acima, foi possível aplicar conhecimentos já existentes na biblioteca \textbf{React}, biblioteca esta usada para o desenvolvimento do projeto, onde permitiu aplicar novos conhecimentos sobre a mesma. O uso de \textit{\textbf{Styled Components}} foi um dos conhecimentos adquiridos nestas formações. O uso de \textit{\textbf{Styled Components}} permite realizar o estilo do projeto sem ser necessário criar ficheiros \texttt{.css}, ou mesmo \texttt{.sass}, recorrendo para tal a \textbf{JavaScript} ou \textbf{TypeScript} e a recorrendo a \textit{template literals} para escrever \textbf{\glsShortUnder{css}}. Desta forma é possível introduzir variáveis passadas através de argumento para o estilo.

Outra das formações realizadas permitiu conhecer melhor o uso de \textbf{GraphQL} para a construção de uma \textbf{\glsShortUnder{api}}, apesar de não ser o foco deste projeto.

% ==> Future Work <== %
\section{Trabalhos Futuros}

Ao longo do projeto foram surgindo formas de melhorar o projeto, bem como pontos que seriam importantes abordar, porém o tempo era limitado, não sendo possível avaliar o esforço que implicaria realizar algumas destas e posteriormente implementar as mesmas. Nos paragráfos seguintes é possível encontrar alguns dos trabalhos que poderiam ser realizados futuramente de forma a melhorar o projeto na sua totalidade.

\subsection{Biblioteca de Componentes}

Um dos pontos seria a criação de uma biblioteca de componentes, isto deve-se essencialmente a existerem componentes iguais em ambas as componentes (\textit{backoffice} e \textit{frontoffice}), evitando também assim o termo \textbf{\textit{\glsShortUnder{dry}}}, ou seja, usar ``pedaços'' de código iguais, mas em vários locais.

A criação desta biblioteca facilitaria o uso de componentes comuns a ambas as componentes do projeto (\textit{backoffice} e \textit{frontoffice}), sendo esta instalada através do \textbf{\glsShortUnder{npm}} ou \textbf{Yarn}. Para realizar esta biblioteca existem várias ferramentas, podendo ser criando um projeto \textbf{React}, em que posteriormente este seria publicado como \glslinkUnder{packages}{package}.

\textbf{Referências:} \cite{publishReactPackage,createLibReact}

\subsection{Testes}

Os testes seriam outra das melhorias a implementar, auxiliando na validação do projeto.  Seriam possível realizar dois tipos de testes, os \textit{unit tests} ou testes unitários e ainda testes \textit{end-to-end}.

Nos tópicos que se seguem é possível analisar melhor cada um destes tipos de testes.

\subsubsection{\textit{Unit Testing}}

Os testes unitários, ou \textit{unit tests}, são testes focados em partes isoladas de um projeto ou sistema

\subsubsection{\textit{End-to-end Testing}}

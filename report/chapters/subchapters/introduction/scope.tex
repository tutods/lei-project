\section{Contextualização}

O projeto \textbf{BeApt}, ou \textbf{Be Armada Portuguesa do Trail}, tem como objetivo de preparar atletas para desafios de alta competição, tal como ultramaratonas, ultra-trails, triatlos, entre outros. Desta forma, é necessário o desenvolvimento de uma aplicação \textit{web}, permitindo a sua utilização em qualquer lugar e dispositivo, que permita tanto a atletas como treinadores gerir, visualizar e analisar resultados de uma forma simples e eficaz.

Para tal, foi idealizada uma aplicação \textit{web} composta por duas vertentes, a vertente de \textit{backoffice}, destinada a treinadores e administradores da plataforma e, a vertente de \textit{frontoffice}, destinada aos atletas, onde estes conseguirão ver os treinos que lhe foram atribuídos, bem como registar resultados desses mesmos treinos.

Um dos pontos com grande importância no projeto é a criação de gráficos e \textit{dashboards} de forma a permitir que tanto atletas como treinadores consigam facilmente analisar os valores registados.
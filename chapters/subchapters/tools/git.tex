\section{Controlo de Versões}

Durante o desenvolvimento de todo o projeto foi utilizado o \textbf{GitLab} para controlo de versões, usufruindo de todas as funcionalidades que este oferece. Nos pontos que se seguem será possível analisar toda a parte relativa ao \textbf{GitLab}, desde da criação e atribuição de \textit{issues}, bem como a sua resolução implicando para tal a criação de uma nova \textit{branch} e de um \textit{merge request}.

\subsection{\textit{Board}}

A \textit{board} do \textbf{GitLab} é bastante versátil, permitindo criar \textit{labels} de forma a organizar todas as tarefas, bem como indicar o seu estado atual. Na imagem que se segue é possível analisar a \textit{board} existente para este projeto, bem como as respetivas \textit{labels}.

Na \textit{board} utilizada no decorrer do projeto é possível encontrar as seguintes colunas:

\begin{itemize}
	\item \textbf{\textit{Open}} \textemdash~coluna por defeito do \textbf{GitLab} para todas as \textit{issues} pendentes;
	\item \textbf{\textit{Closed}} \textemdash~coluna por defeito do \textbf{GitLab} para todas as \textit{issues} encerradas/terminadas;
	\item \textbf{\textit{To Do}} \textemdash~coluna com todas as \textit{issues} pendentes;
	\item \textbf{\textit{In Progress}} \textemdash~coluna com todas as \textit{issues} em progresso;
	\item \textbf{\textit{Review}} \textemdash~coluna com todas as \textit{issues} que aguardam análise e futura aprovação.
\end{itemize}

A imagem que se segue apresenta a \textit{board} utilizada no decorrer do projeto.

% TODO add board image

\subsection{\textit{Issues}}

\subsection{\textit{Merge Requests}}
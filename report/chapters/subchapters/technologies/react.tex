\subsection{React}

\begin{minipage}[t]{.3\textwidth}
	\figureFrame{1}{react.png}{\textbf{React} \textemdash~logo}
\end{minipage}
\begin{minipage}[t]{.7\textwidth}
	\minipagerestore
	Existe quem considere que o \textbf{React} é uma \textit{framework} de \textbf{JavaScript}, porém e, ao mesmo tempo, há quem a considere como uma biblioteca de \textbf{JavaScript} baseada em componentes, sendo este o termo correto.

	Os principais objetivos desta biblioteca são essencialmente:

	\begin{itemize}
		\item Fácil Aprendizagem;
		\item Rapidez;
		\item Escalável.
	\end{itemize}
\end{minipage}

\vspace{0.2cm}

Importante referir que em 2020, segundo o \href{https://insights.stackoverflow.com/survey/2020#technology-most-loved-dreaded-and-wanted-web-frameworks-loved2}{StackOverflow}, o \textbf{React} ficou em segundo lugar nas \textit{frameworks} preferidas dos programadores e em primeiro lugar nas mais procuradas.

Em \hyperrefUnder{reactAttachments}{anexo {\footnotesize(página 56 a 70)}} é possível encontrar todas as instruções relativas à criação de um projeto, bem como estrutura de pastas e execução de um projeto \textbf{React}.


\section{Ambiente de Desenvolvimento}

% ==> IDE <== %
\subsection{IDE}

O \glsShortUnder{ide} é a ferramenta com mais destaque no processo de desenvolvimento, visto ser através deste que será escrito todo o código.

No caso do \glsShortUnder{ide} não existe nenhuma obrigatoriedade sobre qual usar, o programador deve escolher qual o \glsShortUnder{ide} com que se identifica mais, conseguindo assim optimizar o seu \textit{workflow}.

\tocSec{\textbf{Visual Studio Code}}
\label{vscodeAttachments}

\tocSubSec{Configurações}

\begin{longlisting}
	\inputminted{json}{code/vscode-settings.json}
	\caption{Configurações utilizadas no \textbf{Visual Studio Code}}
\end{longlisting}

\begin{mybox}{estg}{Nota}
	Para utilizar as Configurações apresentadas devem ser seguidos os passos abaixo:

	\begin{enumerate}
		\item Aceder às configurações do \textbf{Visual Studio Code} no formato \textbf{JSON}, para isso utilizar a tecla de atalho apresentada abaixo de acordo com o sistema operativo e pesquisar pela opção \textit{``Preferences: Open Settings (JSON)''};
		\begin{itemize}
			\item \textbf{No macOS:} \texttt{CMD + SHIFT + P};
			\item \textbf{No Windows/Linux:} \texttt{CTRL + SHIFT + P}.
		\end{itemize}
		\item Copiar as configurações apresentadas e colar no ficheiro \texttt{settings.json} (ficheiro que abriu no passo anterior).
		\begin{itemize}
			\item \textbf{Nota:} caso já possua configurações neste ficheiro, basta remover as chavetas inicias (\verb|{}|) no código apresentado e colocar as restantes configurações.
		\end{itemize}
	\end{enumerate}
\end{mybox}

\tocSubSec{Extensões}

Como referido anteriormente, o \textbf{Visual Studio Code} é rico em extensões, tornando-o bastante versátil e capaz de ser utilizado para qualquer linguagem ou finalidade. Abaixo são apresentadas algumas das extensões usadas no desenvolvimento deste projeto.

\begin{minipage}[t]{0.5\textwidth}
	\centering
	\figureFrame{1}{es7-react.png}{Extensão \textbf{ES7 React/Redux/GraphQL/React-Native snippets}}

	\href{https://marketplace.visualstudio.com/items?itemName=dsznajder.es7-react-js-snippets}{Link}
\end{minipage}
\begin{minipage}[t]{0.5\textwidth}
	\centering
	\figureFrame{1}{auto-import.png}{Extensão \textbf{Auto Import}}

	\href{https://marketplace.visualstudio.com/items?itemName=steoates.autoimport}{Link}
\end{minipage}

\vspace{0.25cm}

\begin{minipage}[t]{0.5\textwidth}
	\centering
	\figureFrame{1}{auto-close-tag.png}{Extensão \textbf{Auto Close Tag}}

	\href{https://marketplace.visualstudio.com/items?itemName=formulahendry.auto-close-tag}{Link}
\end{minipage}
\begin{minipage}[t]{0.5\textwidth}
	\centering
	\figureFrame{1}{auto-rename.png}{Extensão \textbf{Auto Rename Tag}}

	\href{https://marketplace.visualstudio.com/items?itemName=formulahendry.auto-rename-tag}{Link}
\end{minipage}

\vspace{0.25cm}

\begin{minipage}[t]{0.5\textwidth}
	\centering
	\figureFrame{1}{es-lint.png}{Extensão \textbf{ESLint}}

	\href{https://marketplace.visualstudio.com/items?itemName=dbaeumer.vscode-eslint}{Link}
\end{minipage}
\begin{minipage}[t]{0.5\textwidth}
	\centering
	\figureFrame{1}{sass-extension.png}{Extensão \textbf{Sass}}

	\href{https://marketplace.visualstudio.com/items?itemName=Syler.sass-indented}{Link}
\end{minipage}

\begin{mybox}{estg}{Nota}
	As extensões apresentadas têm apenas a finalidade oferecer mais funcionalidades ou \textit{snippets} ao \glsShortUnder{ide} em questão, o \textbf{Visual Studio Code}.
\end{mybox}
\subsubsection{WebStorm}

\begin{minipage}{.3\textwidth}
	\figureFrame{.45}{webstorm.png}{\textbf{WebStorm} \textemdash~logo}
\end{minipage}
\begin{minipage}{.7\textwidth}
	\minipagerestore
	O \textbf{WebStorm} é outro \glsShortUnder{ide} bastante conhecido e ``poderoso'', não sendo necessário instalar \textit{plugins}/extensões devido a este ser bastante completo.

	Este \glsShortUnder{ide} faz parte das muitas ferramentas disponibilizadas pela \textbf{JetBrains}, tendo como principal vantagem a capacidade de \textit{\glslinkUnder{autocomplete}{autocomplete}} sem a necessidade de \textit{plugins}/extensões adicionais.

\end{minipage}



% ==> Code Formater <== %
\subsection{Prettier}

\begin{minipage}{.3\textwidth}
	\figureFrame{.45}{prettier.png}{\textbf{Prettier} \textemdash~logo}
\end{minipage}
\begin{minipage}{.7\textwidth}
	\minipagerestore
	O \textbf{Prettier} é um \glslinkUnder{packages}{package} destinado à formatação do código auxiliando os desenvolvedores durante todo o processo de desenvolvimento, permitindo criar um ficheiro de configuração com todas as regras que serão aplicadas a quando a formatação do código.

	O \textbf{Prettier} suporta linguagens/\textit{frameworks} como:

	\begin{itemize}
		\item JavaScript;
		\item JSX;
		\item Angular;
		\item Vue;
		\item TypeScript;
		\item CSS, Less e SCSS;
		\item HTML;
		\item JSON;
		\item Markdown;
		\item YAML;
		\item GraphQL;
		\item Entre outras.
	\end{itemize}


	\href{https://prettier.io}{Site Oficial}
\end{minipage}



% ==> Package Manager <== %
\subsection{Gestor de Pacotes}

Como gestor de pacotes, ou \textit{package manager}, podem ser utilizadas duas soluções, sendo elas o \textbf{\glsShortUnder{npm}} e o \textbf{Yarn}. Ambos possuem o mesmo objetivo, a gestão de pacotes num projeto, sendo que o \textbf{\glsShortUnder{npm}} vem incluso na instalação no \textbf{NodeJS}, já por sua vez o \textbf{Yarn} necessita de ser instalado posteriormente.

O \textbf{Yarn} conta com algumas melhorias em relação ao \textbf{\glsShortUnder{npm}}, na tabela que se segue é possível analisar uma pequena comparação entre ambos\footnote{Retirado de \cite{yarnVSNpm}}.

\begin{table}[h!]
	\renewcommand{\arraystretch}{1.25}
	\centering
	\begin{tabularx}{.85\textwidth}{ |c X X X| }
		\rowcolor{estg} & {\color[HTML]{FFFFFF} \textbf{Sem Cache}} & 	{\color[HTML]{FFFFFF} \textbf{Com Cache}} & {\color[HTML]{FFFFFF} \textbf{Reinstalar}} \\\hline


		\textbf{NPM 6.13.4} & 67 segundos & 61 segundos & 28 segundos \\\hline
		\textbf{Yarn 1.21.1} & 57 segundos & 29 segundos & 1.2 segundos \\\hline
	\end{tabularx}

	\caption{Comparação entre \textbf{Yarn} e \textbf{NPM}}
\end{table}

Além das diferenças apresentadas acima, o \textbf{Yarn} conta com outras melhorias em comparação ao \textbf{\glsShortUnder{npm}}, como por exemplo:

\begin{itemize}
	\item Interface mais \textit{clean};
	\item Facilidade de uso \textemdash ~determinados comandos tornam-se mais intuitivos com o \textbf{Yarn};
	\item Possibilidade de reinstalar \glslinkUnder{packages}{packages} sem conexão à Internet.
\end{itemize}


Em \hyperrefUnder{yarnAttachments}{anexo {\footnotesize (página 79 e 80)}} é possível encontrar como proceder à instalação do \textbf{Yarn}.

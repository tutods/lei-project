\tocSec{\textbf{NodeJs}}
\label{nodeAttachments}

\tocSubSec{Instalação}

\tocSubSubSec{macOS}

A instalação do \textbf{NodeJS} no macOS pode ser realizada de diversas formas, as que são apresentas de seguida são apenas algumas das soluções existentes.
\begin{itemize}
	\item \textbf{Via HomeBrew:} \verb|brew install node|;
	\item \textbf{Via \glsShortUnder{nvm}:} \verb|nvm install --lts|
\end{itemize}

De referir que o\textbf{\glsShortUnder{nvm}}\footnote{\href{https://github.com/nvm-sh/nvm}{Repositório oficial do NVM}} é uma ferramenta bastante útil para os utilizadores que necessitam de mais do que uma versão do \textbf{NodeJS}, conseguindo posteriormente realizar a atribuição de uma versão específica a cada projeto, recorrendo para tal ao comando \verb|nvm use <version>|

\tocSubSubSec{Linux}

Para realizar a instalação do \textbf{NodeJS} em Linux é\footnote{\textbf{Nota:} os comandos apresentados são para distribuições baseadas em Debian} bastante simples, recorrendo para tal ao comando \verb|sudo apt install nodejs|, ou novamente, recorrendo ao \textbf{\glsShortUnder{nvm}} como foi apresentado anteriormente.

\tocSubSubSec{Windows}

Para realizar a instalação do \textbf{NodeJS} no Windows basta fazer o download do instalador (ficheiro \verb|.exe|) e seguir os passos apresentados, a imagem que se segue apresenta o ecrã inicial da instalação do \textbf{NodeJS}.

\figureFrame{0.5}{node-install-windows.png}{Ecrã inicial da instalação do \textbf{NodeJS} no Windows}

\tocSubSec{Arquitetura}
\label{nodeArch}

\figureFrame{.7}{node-architecture.png}{Arquitetura do \textbf{NodeJS} em comparação com a arquitetura tradicional}
\section{Formações Adicionais}

De forma a conseguir adquirir novos e aprofundar conhecimentos, foram realizadas algumas formações adicionais, complementares ao estágio (por iniciativa própria). A lista que se segue apresenta projetos realizados durante estas formações, bem como as respetivas ligações para o código desenvolvido nestas.

\clearpage

\begin{minipage}[t]{0.45\textwidth}
	\begin{itemize}
		\item \textbf{Next Level Week 04}
			\begin{itemize}
				\item \href{http://nextlevelweek.com/}{Informações};
				\item \textbf{Repositório:} \href{https://github.com/TutoDS/nlw04-react}{GitHub};
				\item \textbf{\textit{Deploy}:} \href{https://move-it-tutods.vercel.app}{Vercel};
			\end{itemize}

		\item \textbf{\textit{ReactJS Challenge~\textemdash~Slack Clone}}
			\begin{itemize}
				\item \href{https://www.youtube.com/channel/UCqrILQNl5Ed9Dz6CGMyvMTQ}{Canal do YouTube}
				\item \textbf{Repositório:} \href{https://github.com/TutoDS/reactjs-slack-clone-challenge}{GitHub};
				\item \textbf{\textit{Deploy}:} \href{https://slack-clone-challenge-c35ca.web.app/}{Firebase};
			\end{itemize}

			\item \textbf{Twitter \textit{UI Clone}}
			\begin{itemize}
				\item \href{https://www.youtube.com/watch?v=K-8z_4xvT3o}{Vídeo}
				\item \textbf{Repositório:} \href{https://github.com/TutoDS/twitter-ui-clone}{GitHub};
				\item \textbf{\textit{Deploy}:} \href{https://twitter-clone-tutods.netlify.app/}{Netlify}
			\end{itemize}
	\end{itemize}
\end{minipage}
\begin{minipage}[t]{0.45\textwidth}
	\begin{itemize}
		\item \textbf{Next Level Week 05}
			\begin{itemize}
				\item \href{http://nextlevelweek.com/}{Informações};
				\item \textbf{Repositório:} \href{https://github.com/TutoDS/nlw05-react}{GitHub};
				\item \textbf{\textit{Deploy}:} \href{https://podcastr-tutods.vercel.app/}{Vercel}
			\end{itemize}

		\item \textbf{LinkedIn \textit{UI Clone}}
			\begin{itemize}
				\item \href{https://www.youtube.com/watch?v=xP3cxbDUtrc}{Vídeo}
				\item \textbf{Repositório:} \href{https://github.com/TutoDS/reactjs-linkedin-clone}{GitHub}
				\item Por Concluir
			\end{itemize}

		\item \textbf{TypeGraphQL} - Code/drop \#74
			\begin{itemize}
				\item \href{https://www.youtube.com/watch?v=qMc5A5-Ktuw}{Vídeo}
				\item \textbf{Repositório:} \href{https://github.com/TutoDS/typegraphql-code-drops-74}{GitHub}
				\item \textbf{Temas Principais:} GraphQL e TypeGraphQL
			\end{itemize}
	\end{itemize}
\end{minipage}

\vspace{10pt}

Nos projetos apresentados acima, foi possível aplicar conhecimentos já existentes na biblioteca \textbf{React}, biblioteca esta usada para o desenvolvimento do projeto, onde permitiu aplicar novos conhecimentos sobre a mesma. O uso de \textit{\textbf{Styled Components}} foi um dos conhecimentos adquiridos nestas formações. O uso de \textit{\textbf{Styled Components}} permite realizar o estilo do projeto sem ser necessário criar ficheiros \texttt{.css}, ou mesmo \texttt{.sass}, recorrendo para tal a \textbf{JavaScript} ou \textbf{TypeScript} e a recorrendo a \textit{template literals} para escrever \textbf{\glsShortUnder{css}}. Desta forma é possível introduzir variáveis passadas através de argumento para o estilo.

Outra das formações realizadas permitiu conhecer melhor o uso de \textbf{GraphQL} para a construção de uma \textbf{\glsShortUnder{api}}, apesar de não ser o foco deste projeto.
\section{Trabalhos Futuros}

Ao longo do projeto foram surgindo formas de melhorar o projeto, bem como pontos que seriam importantes abordar, porém o tempo era limitado, não sendo possível avaliar o esforço que implicaria realizar algumas destas e posteriormente implementar as mesmas. Nos paragráfos seguintes é possível encontrar alguns dos trabalhos que poderiam ser realizados futuramente de forma a melhorar o projeto na sua totalidade.

\subsection{Biblioteca de Componentes}

Um dos pontos seria a criação de uma biblioteca de componentes, isto deve-se essencialmente a existerem componentes iguais em ambas as componentes (\textit{backoffice} e \textit{frontoffice}), evitando também assim o termo \textbf{\textit{\glsShortUnder{dry}}}, ou seja, usar ``pedaços'' de código iguais, mas em vários locais.

A criação desta biblioteca facilitaria o uso de componentes comuns a ambas as componentes do projeto (\textit{backoffice} e \textit{frontoffice}), sendo esta instalada através do \textbf{\glsShortUnder{npm}} ou \textbf{Yarn}. Para realizar esta biblioteca existem várias ferramentas, podendo ser criando um projeto \textbf{React}, em que posteriormente este seria publicado como \glslinkUnder{packages}{package}.

\textbf{Referências:} \cite{publishReactPackage,createLibReact}

\subsection{Testes}

Os testes seriam outra das melhorias a implementar, auxiliando na validação do projeto.  Seriam possível realizar dois tipos de testes, os \textit{unit tests} ou testes unitários e ainda testes \textit{end-to-end}.

Nos tópicos que se seguem é possível analisar melhor cada um destes tipos de testes.

\subsubsection{\textit{Unit Testing}}

Os testes unitários, ou \textit{unit tests}, são testes focados em partes isoladas de um projeto ou sistema

\subsubsection{\textit{End-to-end Testing}}

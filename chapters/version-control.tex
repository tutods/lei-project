\chapter{Controlo de Versões}

Durante o desenvolvimento de todo o projeto foi utilizado o \textbf{GitLab} como controlo de versões, usufruindo de todas as funcionalidades que este oferece. Nos pontos que se seguem será possível analisar toda a parte relativa ao \textbf{GitLab}, desde da criação e atribuição de \textit{issues}, bem como a sua resolução implicando para tal a criação de uma nova \textit{branch} e de um \textit{merge request}.

\section{\textit{Board}}

A \textit{board} do \textbf{GitLab} é bastante versátil, permitindo criar \textit{labels} de forma a organizar todas as tarefas, bem como indicar o seu estado atual. Na imagem que se segue é possível analisar a \textit{board} existente para este projeto, bem como as respetivas \textit{labels}.

\section{\textit{Issues}}

\section{\textit{Merge Requests}}
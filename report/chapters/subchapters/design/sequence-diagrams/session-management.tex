\subsection{Gestão de Sessão}
\label{sessionSequenceDiagram}

\figureFrame{1}{sequence/session-management.jpeg}{Diagrama de Sequência de Gestão de Sessão}

A gestão de sessão é um ponto importante a abordar, pois será através da sessão que os atletas, treinadores e administrador poderão ou não executar determinadas ações. O diagrama acima apresenta o fluxo seguindo  na gestão de sessão, sendo que inicialmente é realizado uma inicio de sessão no \textit{front-end} do projeto, que por sua vez vai enviar os dados à \textbf{\glsShortUnder{api}}, onde é verificado se o email introduzido existe na base de dados, no caso de existir é então validada a \textit{password} enviada, sendo comparada com o \textit{hash} (\textit{password} encriptada) armazenado na base de dados. Com todos os dados validados é devolvido um \textbf{\glsShortUnder{jwt} \textit{token}} para utilizar nos pedidos que necessitem de autenticação.

No diagrama é apresentado o caso do pedido da lista de treinos, onde é enviado por \textit{HTTP Headers} o \textit{token}, o excerto de código que se segue apresenta um exemplo de pedido com o \textit{\glslinkUnder{packages}{package}} \textbf{\href{https://github.com/axios/axios}{Axios}} e o envio do \textit{HTTP Header} de autenticação.

\begin{longlisting}
	\begin{minted}[highlightlines={4-6},highlightcolor=yellow!25]{jsx}
		import axios from 'axios';

		axios.get('<api-url>/<route>', {
			headers: {
				Authorization: 'Bearer <jwt-token>'
			}
		});
	\end{minted}

	\caption{Exemplo de pedido com o package \textbf{Axios} e autenticação por \textbf{HTTP Headers}}
\end{longlisting}

Assim, o \textit{token} \textbf{\glsShortUnder{jwt}} recebido é enviado para a \textbf{\glsShortUnder{api}}, onde será validado se este expirou ou não. No caso deste ter expirado, o cliente será redirecionado para a página de inicio de sessão com o \textit{HTTP Status Code} 401\footnote{\textbf{HTTP Status Code 401 \textemdash} ~\textit{Unauthorized} (\href{https://developer.mozilla.org/pt-BR/docs/Web/HTTP/Status/401}{Mais Informações})} na resposta da \textbf{\glsShortUnder{api}}.
\subsection{TypeScript}

\begin{minipage}[t]{.3\textwidth}
	% ==> Figure Frame Function <== %
	\figureFrame{.5}{typescript.png}{\textbf{TypeScript} \textemdash~logo}
\end{minipage}
\begin{minipage}[t]{.7\textwidth}
	\minipagerestore

	O \textbf{TypeScript} é uma das tecnologias que é possível encontrar neste projeto tanto em \glslinkUnder{frontend}{\textit{front-end}} como \glslinkUnder{backend}{\textit{back-end}}.

	O \textbf{TypeScript}, segundo a própria \textbf{Microsoft} (detentora do \textbf{TypeScript}, é nada mais nada menos do que \textbf{JavaScript}, porém com a adição de tipos.

	\begin{quotebox}
		``TypeScript extends JavaScript by adding types.''
		\tcblower
		Retirado do \href{https://www.typescriptlang.org}{\textit{website} oficial}
	\end{quotebox}

\end{minipage}

\vspace{0.2cm}

Em 2020, o \textbf{TypeScript} ficou em segundo lugar das linguagens preferidas e, em quarto lugar das linguagens mais procuradas, dados do \href{https://insights.stackoverflow.com/survey/2020#technology-most-loved-dreaded-and-wanted-languages-loved}{\textbf{StackOverflow}}.

Devido a esta tipagem que é adicionada, o código torna-se mais facilmente interpretado, facilitando também o processo de \textit{debug}, bem como as validações realizadas no processo de \textit{build}. O excerto de código abaixo, retirado do \href{https://www.typescriptlang.org}{\textit{website} oficial}, tem como objetivo demonstrar a validação que é realizada pelo \textbf{TypeScript}.

\begin{longlisting}
	\begin{minted}[highlightlines={7},highlightcolor=red!35]{js}
		const user = {
			firstName: "Angela",
			lastName: "Davis",
			role: "Professor"
		}

		console.log(user.name)
	\end{minted}

	\caption{Excerto de código com validação \textbf{TypeScript}}
\end{longlisting}

No caso, a linha 7 (assinalada com a cor vermelha), irá causar a mensagem de erro abaixo que indica que a propriedade \texttt{name} não existe no objeto \texttt{user}.

\begin{errorbox}{Erro Apresentado}
	Property \texttt{'name'} does not exist on type '\mintinline{js}{{ firstName: string; lastName: string; role: string; }}'.
\end{errorbox}

A tabela apresentada demonstra as principais diferenças entre o \textbf{TypeScript} e o \textbf{JavaScript}.

\begin{table}[h!]
	\renewcommand{\arraystretch}{1.25}
	\begin{tabularx}{\textwidth}{ |X|X| }
		\rowcolor{estg}	{\color[HTML]{FFFFFF} \textbf{TypeScript}} & {\color[HTML]{FFFFFF} \textbf{JavaScript}} \\
		Linguagem orientada a objetos & Linguagem de \textit{Scripting} \\\hline
		Possui tipagem estática & Não possui tipagem \\\hline
		Suporte a módulos & Sem suporte a módulos\\\hline
		Possui suporte a definição de parâmetros opcionais em funções & Não suporta a definição de parâmetros opcionais em funções \\\hline
	\end{tabularx}

	\caption{Principais diferenças entre \textbf{TypeScript} e \textbf{JavaScript}}
\end{table}

Em \hyperrefUnder{typescriptAttachment}{anexo {\footnotesize(página 53 a 55)}} é possível encontrar como realizar a instalação do \textbf{TypeScript} e ainda, um exemplo de uma configuração realizada através do ficheiro \texttt{tsconfig.json}.
